% ===================================================
\chapter{Background}
% ===================================================

% % ===================================================
% \section{Robotics Pipelines}
% % ===================================================
% 
% % ---------------------------------------------------
% \subsection{Robotic Operating System}
% % ---------------------------------------------------
% 
% % ---------------------------------------------------
% \subsection{Robotic Operating System}
% % ---------------------------------------------------
% 
% % ===================================================
% \section{Modeling}
% % ===================================================
% 
% % ---------------------------------------------------
% \subsection{Classification}
% % ---------------------------------------------------
% Modeling 
% 
% % ---------------------------------------------------
% \subsection{Bayesian Stuff}
% % ---------------------------------------------------
% Chiu-Liu Trees
% 
% % ---------------------------------------------------
% \subsection{Markov Chains, MDPs, POMDPs}
% % ---------------------------------------------------
% 
% % ---------------------------------------------------
% \subsection{Reinforcement Learning}
% % ---------------------------------------------------
% 
% % ---------------------------------------------------
% \subsection{Path Planning}
% % ---------------------------------------------------
% 
% % ---------------------------------------------------
% \subsection{Bandit Problems}
% % ---------------------------------------------------
% 
% % ---------------------------------------------------
% \subsection{Feasibility}
% % ---------------------------------------------------
% 
% % ---------------------------------------------------
% \subsection{Game Theory}
% % ---------------------------------------------------
% 
% % ---------------------------------------------------
% \subsection{Control}
% % ---------------------------------------------------
% 
% % ---------------------------------------------------
% \subsection{Rotations}
% % ---------------------------------------------------
% 
% % ---------------------------------------------------
% \subsection{Distance Transforms and Geodesics}
% % ---------------------------------------------------

% ===================================================
\section{Notation for Traditional Path Planning}
% ===================================================

\theoremstyle{definition}
\begin{definition}{Maximal Reachable Set}
A fibration is a mapping between two topological spaces that has the homotopy lifting property for every space $X$.
\end{definition}

Taken from page 17 of \cite{lavalle2006planning}:

\begin{definition}{State}
Temp
\end{definition}

\begin{definition}{state space $X$}
Finite or countably infinite set of states.
\end{definition}

\begin{definition}{Initial State $x_1 \in X$}
Temp
\end{definition}

\begin{definition}{Goal Set $X_G \subset X$}
Temp
\end{definition}

\begin{definition}{Time}
Temp
\end{definition}

\begin{definition}{Action $u$}
Temp
\end{definition}

\begin{definition}{action space $U(x)$}
A finite set of actions for each state $x \in X$
\end{definition}

\begin{definition}{(state transition function)}
 $f(x,u) \in X$
A finite set of actions for each state $x \in X$
\end{definition}

\begin{definition}{Feasibility Criterion}
Temp
\end{definition}

\begin{definition}{Optimality Criterion}
Temp
\end{definition}

\begin{definition}{Plan}
Temp
\end{definition}

\begin{definition}{Potential Function}
Temp
\end{definition}

\begin{definition}{Configuration Space}
Temp
\end{definition}

\begin{definition}{XYZ Space}
Temp
\end{definition}

\begin{definition}{XY Space $M$}
Temp
\end{definition}


% % ===================================================
% \chapter{Related Work}
% % ===================================================
% % show my problem is unsolved
% % • show my problem is interesting
% % • show my idea works
% % • show how my idea compares to other ideas
% % • No related work yet! Include at end.
% % – Problem 1: describing alternative approaches gets between the reader
% % and your idea (I feel tired)
% % – Problem 2: the reader knows nothing about the problem yet; so your
% % (carefully trimmed) description of various technical tradeoffs is abso-
% % lutely incomprehensible (I feel stupid)
% % – instead, Concentrate single-mindedly on a narrative that
% % ∗ Describes the problem, and why it is interesting
% % ∗ Describes your idea
% % ∗ Defends your idea, showing how it solves the problem, and filling
% % out the details
% % ∗ On the way, cite relevant work in passing, but defer discussion to
% % the end
% % • show how my idea compares to other ideas
% % • Be generous to the competition. In his inspiring paper [Foo98] Foogle shows....
% % We develop his foundation in the following ways...
% % • Giving credit to others does not diminish the credit you get from your paper
% % • Acknowledge weaknesses in your approach
% % • Failing to give credit to others can kill your paper: You don’t know that it’s
% % an old idea (bad) or You do know, but are pretending it’s yours (very bad)
% 
% % ====================
% \section{Parking Lot Detection}
% % ====================
% % Wang \cite{wang2014automatic}
% % Jung \cite{jung2014semiautomatic}
% % 
% % Pouria's thesis \cite{talebifard2014risk}
% % Talk: \url{https://www.youtube.com/watch?v=Db7EhLYf-Yk#t=1h11m07s}
% % 
% % 
% % file:papers/jiang1999sensor.pdf
% % Basic old paper on parallel parking. Uses ultrasonic sensor, then
% % fixed planning.
% % 
% % file:papers/fairus2011development.pdf
% % Old paper. Scanning methods.
% % 
% % file:papers/wang2013automatic.pdf
% % Newish lit review of automatic parking
% % 
% % file:papers/abad2007parking.pdf
% % Parking space detection using 3d vision. Parallel parking
% % 
% % file:papers/lalonde2012single.pdf
% % CVPR workshop paper.
% % 
% % file:papers/suhr2010automatic.pdf
% % Good! Solving my problem, but with a laser rangefinder.
% % found from: http://www.mathworks.com/matlabcentral/answers/126413-how-to-detect-free-spots-in-a-parking-area
% % 
% % file:assets/WheelchairGuide.pdf
% 
% % ====================
% \section{SLAM}
% % ====================
% 
% LSD-Slam and PTAM for monocular seem to be state-of-art: \url{http://vision.in.tum.de/research/lsdslam}
% 
% review paper: \url{http://www-personal.acfr.usyd.edu.au/tbailey/papers/slamtute2.pdf}
% 
% 2009 review paper: \url{http://www.le2i.cnrs.fr/IMG/publications/2172_Muhammad_EI_2009.pdf}
% 
% % ====================
% \section{Collision Avoidance}
% % ====================
% Old paper using quadtrees\cite{ghoshray1996comprehensive}
% 
% Bigdog \cite{raibert2008bigdog}
% 
% Springer handbook of robotics \cite{siciliano2008springer}: Vector field histograms, as described in Pouria's thesis \cite{talebifard2014risk}
% 
% 
% % ====================
% \section{Time Optimal Trajectory Planning}
% % ====================
% \cite{fiorini1996time}
