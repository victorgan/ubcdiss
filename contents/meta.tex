% ====================
\chapter{Metachapter: How to Do Research}
% ====================

% ====================
\section{How to do research}
% ====================
Todo: make into checklist form: How to do research \cite{chapman1988how}
\cite{hwang2008how}
\cite{desjardins2008how}

Chapman in 1988~\cite{chapman1988how}: informal rules of thumb advice. Each
section corresponds to a chapter in his document, in the same order.


% ====================
\section{How to read}
% ====================

Chapman~\cite{chapman1988how}:
\blockquote{Many researchers spend more than half their time reading}

\blockquote{The time to start reading is now. Once you start seriously working
on your thesis you’ll have less time, and your reading will have to be more
focused on the topic area. During your first two years, you’ll mostly be doing
class work and getting up to speed on AI in general. For this it suffices to
read textbooks and published journal articles.}

\blockquote{The reading lists for the AI qualifying exams at other
universities—particularly Stanford—are also useful}

\blockquote{There are three phases to reading. The
first is to see if there’s anything of interest in it at all. Second is to find
the part of the paper that has the good stuff. Most fifteen page papers could
profitably be rewritten as one-page papers; you need to look for the page that
has the exciting stuff. Often this is hidden somewhere unlikely. What the author
finds interesting about his work may not be interesting to you, and vice versa.
Finally, you may go back and read the whole paper through if it seems
worthwhile}

\blockquote{Read with a question in mind. “How can I use this?” “Does this really do
what the author claims?” “What if...?” Understanding what result has been presented
is not the same as understanding the paper. Most of the understanding
is in figuring out the motivations, the choices the authors made (many of them
implicit), whether the assumptions and formalizations are realistic, what directions
the work suggests, the problems lying just over the horizon, the patterns
of difficulty that keep coming up in the author’s research program, the political
points the paper may be aimed at, and so forth.}

\blockquote{If you are interested in an area and read a few papers about it, try
implementing toy versions of the programs being described. This gives you a more
concrete understanding.}

% --------------------
\subsection{Reading a paper checklist}
% --------------------

\begin{itemize}
\item How can I use this?
\item Does this really do what the author claims?
\item Are assumptions and formalizations realistic?
\item Implement a toy version of the program
\end{itemize}

% ====================
\section{Getting connected}
% ====================

Chapman~\cite{chapman1988how}:
\blockquote{it’s important to get plugged into the Secret Paper Passing Network.
This informal organization is where all the action in AI really is.}

This can be done by
\begin{itemize}
\item Mailing lists
\item Talk about an idea you’ve had with someone who knows the
field; they are likely to say, “Have you read X?” 
\item When you read a paper that excites you, send it to people who might be
interested in it. They'll return the favor.
\item Discussion groups
\item Read papers on people's desks, filing cabinets
\item Distribute draft copies of things you write (``Please do not photocopy or
quote" on the front page). Most people don’t read most of the papers they’re
given, so don’t take it personally
\item When you finish a paper, send copies to everyone you think might be
interested.
\item Make yourself the bridge between two groups of
interesting people working on related problems who aren’t talking to each
other
\item Keep a log of interesting references.
\item Try talking to them about the really good or unbelievably foolish stuff
you’ve been reading. Hang out, dinners, lunches, informal gatherings.
\item Get a business card.
\item At conferences, walk up to someone whose paper you’ve liked, say ``I really
liked your paper", and ask a question.
\item Get summer jobs away at other labs.
\end{itemize}

% ====================
\section{Learning other fields}
% ====================
\begin{itemize}
\item Take a graduate course. This is solidest, but is often not efficient.
\item Read a textbook. Not a bad approach, but textbooks are usually out of date, and generally have a high ratio of words to content.
\item Find out what the best journal in the field is, maybe by talking to someone who knows about it. Then skim the last few years worth and follow the reference trees. This is usually the fastest way to get a feel of what is happening, but can give you a somewhat warped view.
\item Find out who’s most famous in the field and read their books.
\item Hang out with grad students in the field.
\item Go to talks. 
\end{itemize}

% ====================
\section{Keeping a Research Notebook}
% ====================
\blockquote{Record in your notebook ideas as they come up. Nobody except you is going
to read it, so you can be random. Put in speculations, current problems in your
work, possible solutions. Work through possible solutions there. Summarize for
future reference interesting things you read.}


\begin{itemize}
\item HTML Website. Pros: easily accessed from the web. Cons: Hard to update
\item One big Latex file. Pros: easy to transfer to a paper. Cons: No hyperlinks.
\end{itemize}

% ====================
\section{Writing}
% ====================
Print this off and use this as a checklist whenever I write a paper.

Stuff from Simon Peyton-Jones' talk~\cite{jones2013how}.
% --------------------
\subsection{Abstract}
% --------------------
\begin{itemize}
\item Four sentences:
\item state the problem
\item state why it's an interesting problem
\item say what your solution achieves
\item say what follows from your solution
\end{itemize}

% --------------------
\subsection{Intro}
% --------------------
\begin{itemize}
\item (1 page in conference)
\item Describe the problem
\item State your contributions (write the list of contributions fist)
\item that's all
\item use an example to introduce the problem
\item Reader thinks``gosh, if they can really deliver this, that'd be exciting; I’d better read on" \cite{jones2013how}
\item Contributions should be refutable
\item No "the rest of the paper is". Instead, use forward references from the narrative in the introduction. The introduction (including the contributions) should survey the whole paper, and therefore forward reference every important part
\end{itemize}

% --------------------
\subsection{The Problem}
% --------------------
\begin{itemize}
\item 1 page
\end{itemize}

% --------------------
\subsection{Related Work Purpose (1-2 pages)}
% --------------------
\begin{itemize}
\item show my problem is unsolved
\item show my problem is interesting
\item show my idea works
\item show how my idea compares to other ideas
\end{itemize}

\begin{itemize}
\item No related work yet! Include at end.
\begin{itemize}
    \item Problem 1: describing alternative approaches gets between the reader and your idea (I feel tired)
    \item Problem 2: the reader knows nothing
    about the problem yet; so your (carefully
    trimmed) description of various technical
    tradeoffs is absolutely incomprehensible (I feel stupid)
    \item instead, Concentrate single-mindedly on a narrative that
    \begin{itemize}
        \item Describes the problem, and why it is interesting
        \item Describes your idea
        \item Defends your idea, showing how it solves the problem,
and filling out the details
        \item On the way, cite relevant work in passing, but defer
discussion to the end
    \end{itemize}
\end{itemize}

\item show how my idea compares to other ideas
\item Be generous to the competition. “In his inspiring paper
[Foo98] Foogle shows.... We develop his foundation in the
following ways...
\item Giving credit to others does not diminish
the credit you get from your paper
\item Acknowledge weaknesses in your approach
\item Failing to give credit to others can kill
your paper: You don't know that it's an old idea (bad) or You do know, but are pretending it's yours (very bad)

\item 

\end{itemize}

% --------------------
\subsection{Method}
% --------------------
\begin{itemize}
\item In a paper you MUST provide the details,
but FIRST convey the idea
\item Introduce the problem, and your idea, using
EXAMPLES and only then present the general case
\item Explain it as if you were speaking to someone using
a whiteboard
\item Conveying the intuition is primary, not secondary
\item Once your reader has the intuition, she can follow
the details (but not vice versa)
\item Even if she skips the details, she still takes away
something valuable
\end{itemize}

Evidence
\begin{itemize}
\item Your introduction makes claims; The body of the paper provides evidence to support each claim
\item Check each claim in the introduction, identify the
evidence, and forward-reference it from the claim
\item Evidence can be: analysis and comparison, theorems,
measurements, case studies
\end{itemize}


% ====================
\section{Giving Talks}
% ====================

% ====================
\section{Programming}
% ====================

% ====================
\section{How to Select a Graduate Advisor}
% ====================

% ====================
\section{How to Choose a Thesis Topic and Avoid Wasting Time}
% ====================

% ====================
\section{How to do Resarch Methodology}
% ====================

% ====================
\section{Emotional Factors in the Research Process}
% ====================
