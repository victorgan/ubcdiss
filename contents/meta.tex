% ====================
\chapter{Metachapter: How to Do Research}
% ====================

% ====================
\section{How to do research}
% ====================
Todo: make into checklist form: How to do research \cite{chapman1988how}
\cite{hwang2008how}
\cite{desjardins2008how}

Chapman in 1988~\cite{chapman1988how}: informal rules of thumb advice. Each
section corresponds to a chapter in his document, in the same order.


% ====================
\section{How to read}
% ====================

Chapman~\cite{chapman1988how}:
\blockquote{Many researchers spend more than half their time reading}

\blockquote{The time to start reading is now. Once you start seriously working
on your thesis you’ll have less time, and your reading will have to be more
focused on the topic area. During your first two years, you’ll mostly be doing
class work and getting up to speed on AI in general. For this it suffices to
read textbooks and published journal articles.}

\blockquote{The reading lists for the AI qualifying exams at other
universities—particularly Stanford—are also useful}

\blockquote{There are three phases to reading. The
first is to see if there’s anything of interest in it at all. Second is to find
the part of the paper that has the good stuff. Most fifteen page papers could
profitably be rewritten as one-page papers; you need to look for the page that
has the exciting stuff. Often this is hidden somewhere unlikely. What the author
finds interesting about his work may not be interesting to you, and vice versa.
Finally, you may go back and read the whole paper through if it seems
worthwhile}

\blockquote{Read with a question in mind. “How can I use this?” “Does this really do
what the author claims?” “What if...?” Understanding what result has been presented
is not the same as understanding the paper. Most of the understanding
is in figuring out the motivations, the choices the authors made (many of them
implicit), whether the assumptions and formalizations are realistic, what directions
the work suggests, the problems lying just over the horizon, the patterns
of difficulty that keep coming up in the author’s research program, the political
points the paper may be aimed at, and so forth.}

\blockquote{If you are interested in an area and read a few papers about it, try
implementing toy versions of the programs being described. This gives you a more
concrete understanding.}

% --------------------
\subsection{Reading a paper checklist}
% --------------------

\begin{itemize}
\item How can I use this?
\item Does this really do what the author claims?
\item Are assumptions and formalizations realistic?
\item Implement a toy version of the program
\end{itemize}

% ====================
\section{Getting connected}
% ====================

Chapman~\cite{chapman1988how}:
\blockquote{it’s important to get plugged into the Secret Paper Passing Network.
This informal organization is where all the action in AI really is.}

This can be done by
\begin{itemize}
\item Mailing lists
\item Talk about an idea you’ve had with someone who knows the
field; they are likely to say, “Have you read X?” 
\item When you read a paper that excites you, send it to people who might be
interested in it. They'll return the favor.
\item Discussion groups
\item Read papers on people's desks, filing cabinets
\item Distribute draft copies of things you write (``Please do not photocopy or
quote" on the front page). Most people don’t read most of the papers they’re
given, so don’t take it personally
\item When you finish a paper, send copies to everyone you think might be
interested.
\item Make yourself the bridge between two groups of
interesting people working on related problems who aren’t talking to each
other
\item Keep a log of interesting references.
\item Try talking to them about the really good or unbelievably foolish stuff
you’ve been reading. Hang out, dinners, lunches, informal gatherings.
\item Get a business card.
\item At conferences, walk up to someone whose paper you’ve liked, say ``I really
liked your paper", and ask a question.
\item Get summer jobs away at other labs.
\end{itemize}

\blockquote{It’s important to get a variety of people who will regularly review
your work, because it's very easy to mislead yourself (and often your advisor as
well) into thinking you are making progress when you are not, and so zoom off
into outer space.}

% ====================
\section{Learning other fields}
% ====================
\begin{itemize}
\item Take a graduate course. This is solidest, but is often not efficient.
\item Read a textbook. Not a bad approach, but textbooks are usually out of date, and generally have a high ratio of words to content.
\item Find out what the best journal in the field is, maybe by talking to someone who knows about it. Then skim the last few years worth and follow the reference trees. This is usually the fastest way to get a feel of what is happening, but can give you a somewhat warped view.
\item Find out who’s most famous in the field and read their books.
\item Hang out with grad students in the field.
\item Go to talks. 
\end{itemize}

% ====================
\section{Keeping a Research Notebook}
% ====================
\blockquote{Record in your notebook ideas as they come up. Nobody except you is going
to read it, so you can be random. Put in speculations, current problems in your
work, possible solutions. Work through possible solutions there. Summarize for
future reference interesting things you read.}

Methods?
\begin{itemize}
\item HTML Website. Pros: easily accessed from the web. Cons: Hard to update
\item One big Latex file. Pros: easy to transfer to a paper. Cons: No
hyperlinks. I'll use this.
\end{itemize}

Vera Johnson-Steiner's book Notebooks of the Mind: Describes how creative
thought emerges from the accumulation of fragmented ideas.

% ====================
\section{Writing}
% ====================

% --------------------
\subsection{General}
% --------------------
\blockquote{Writing down your ideas is the best way to debug them. Usually you will
find that what seemed perfectly clear in your head is in fact an incoherent
mess on paper.

Realize that writing is
a debugging process. Write something sloppy first and go back and fix it up.
Starting sloppy gets the ideas out and gets you into the flow. If you “can’t”
write text, write an outline. Make it more and more detailed until it’s easy
to write the subsubsubsections. If you find it really hard to be sloppy, try
turning the contrast knob on your terminal all the way down so you can’t
see what you are writing. Type whatever comes into your head, even if it
seems like garbage. After you’ve got a lot of text out, turn the knob back
up and edit what you’ve written into something sensible.

Perfectionism can also lead to endless repolishing of a perfectly adequate
paper. This is a waste of time. (It can also be a way of semideliberately
avoiding doing research.) 
}

List of papers that are well-written:
\begin{itemize}
\item None so far!
\end{itemize}

List of poorly written papers/quotes:
\begin{itemize}
\item None so far!
\end{itemize}

General tips:
\begin{itemize}
\item Put the sexy stuff up front, at all levels of organization from paragraph
up to the whole paper.
\item explain why it works and why it’s interesting
\item Do not imitate math texts; their standard of elegance is to say as little
as possible, and so to make the reader’s job as hard as possible. This is not
appropriate for AI.
\item After you have written a paper, delete the first paragraph or the first few
sentences. You’ll probably find that they were content-free generalities, and
that a much better introductory sentence can be found at the end of the
first paragraph of the beginning of the second.
\end{itemize}

% --------------------
\subsection{Goals for Getting Published}
% --------------------
Standards are surprisngly low.
Criteria:
\begin{itemize}
\item Has something new to say
\item Isn't broken in some way
\end{itemize}

Strategies:
\begin{itemize}
\item Try a lot
\item Make sure it's readable
\item Circulate drafts beforehand
\item Read backissues of conference and make sure style/content is appropriate
\item Read the information for authors page specified by conference
\item Read best paper awards
\end{itemize}

% --------------------
\subsection{Abstract}
% --------------------
Print this off and use this as a checklist whenever I write a paper.
Stuff from Simon Peyton-Jones' talk~\cite{jones2013how}.

\begin{itemize}
\item Four sentences:
\item state the problem
\item state why it's an interesting problem
\item say what your solution achieves
\item say what follows from your solution
\end{itemize}

% --------------------
\subsection{Intro}
% --------------------
\begin{itemize}
\item (1 page in conference)
\item Describe the problem
\item State your contributions (write the list of contributions fist)
\item that's all
\item use an example to introduce the problem
\item Reader thinks``gosh, if they can really deliver this, that'd be exciting; I’d better read on" \cite{jones2013how}
\item Contributions should be refutable
\item No "the rest of the paper is". Instead, use forward references from the narrative in the introduction. The introduction (including the contributions) should survey the whole paper, and therefore forward reference every important part
\end{itemize}

% --------------------
\subsection{The Problem}
% --------------------
\begin{itemize}
\item 1 page
\end{itemize}

% --------------------
\subsection{Related Work Purpose (1-2 pages)}
% --------------------
\begin{itemize}
\item show my problem is unsolved
\item show my problem is interesting
\item show my idea works
\item show how my idea compares to other ideas
\end{itemize}

\begin{itemize}
\item No related work yet! Include at end.
\begin{itemize}
    \item Problem 1: describing alternative approaches gets between the reader and your idea (I feel tired)
    \item Problem 2: the reader knows nothing
    about the problem yet; so your (carefully
    trimmed) description of various technical
    tradeoffs is absolutely incomprehensible (I feel stupid)
    \item instead, Concentrate single-mindedly on a narrative that
    \begin{itemize}
        \item Describes the problem, and why it is interesting
        \item Describes your idea
        \item Defends your idea, showing how it solves the problem,
and filling out the details
        \item On the way, cite relevant work in passing, but defer
discussion to the end
    \end{itemize}
\end{itemize}

\item show how my idea compares to other ideas
\item Be generous to the competition. “In his inspiring paper
[Foo98] Foogle shows.... We develop his foundation in the
following ways...
\item Giving credit to others does not diminish
the credit you get from your paper
\item Acknowledge weaknesses in your approach
\item Failing to give credit to others can kill
your paper: You don't know that it's an old idea (bad) or You do know, but are pretending it's yours (very bad)

\item 

\end{itemize}

% --------------------
\subsection{Method}
% --------------------
\begin{itemize}
\item In a paper you MUST provide the details,
but FIRST convey the idea
\item Introduce the problem, and your idea, using
EXAMPLES and only then present the general case
\item Explain it as if you were speaking to someone using
a whiteboard
\item Conveying the intuition is primary, not secondary
\item Once your reader has the intuition, she can follow
the details (but not vice versa)
\item Even if she skips the details, she still takes away
something valuable
\end{itemize}

Evidence
\begin{itemize}
\item Your introduction makes claims; The body of the paper provides evidence to support each claim
\item Check each claim in the introduction, identify the
evidence, and forward-reference it from the claim
\item Evidence can be: analysis and comparison, theorems,
measurements, case studies
\end{itemize}


% ====================
\section{Giving Talks}
% ====================
\blockquote{Since revising a talk is generally much easier than revising a
paper, some people find that this is a good way to find the right way to express
their ideas. (Mike Brady once remarked that all of his best papers started out
as talks.)}

\blockquote{Cornering one of your friends and trying to explain your most recent
brainstorm to him is a good way both to improve your communication skills, and
to debug your ideas.}




% ====================
\section{Programming}
% ==================== 
\blockquote{Like papers, programs can be over-polished. Rewriting code till it’s
perfect, making everything maximally abstract, writing macros and libraries, and
playing with operating system internals has sucked many people out their theses
and out of the field.}

% ====================
\section{How to Select a Graduate Advisor}
% ====================
\begin{itemize}
\item How much direction do you want? 
\item How much contact do you want? 
\item How much pressure do you want? 
\item How much emotional support do you want? 
\item How seriously do you want to take your advisor? 
\item What kind of research group does the advisor provide? 
\item Do you want to be working on a part of a larger project? 
\item Do you want cosupervision? 
\item Is the advisor willing to supervise a thesis on a topic outside his main area of research? 
\item Will the advisor fight the system for you? 
\item Is the advisor willing and able to promote your work at conferences and the like? 
\end{itemize}


\begin{itemize}
\item Read the Lab’s research summary. 
\item Read recent papers of any faculty member whose work seems at all interesting.
\item Talk to as many faculty members as you can during your first semester.
\item Talk to grad students of prospective advisors and ask what working for him or her is like. 
\item Attend faculty member's research group meetings
\end{itemize}

% ====================
\section{How to Choose a Thesis Topic and Avoid Wasting Time}
% ====================
\blockquote{The essential requirement of a Master’s thesis is that it literally
demonstrate mastery: that you have fully understood the state of the art in your
subfield and that you are capable of operating at that level. It is not a
requirement that you extend the state of the art, nor that the Master’s thesis
be publishable.

The actual writing of the PhD thesis generally takes about a year, and an
oft-confirmed rule of thumb is that it will drag on for a year after you are
utterly sick of it.
}

Read other people's theses.

Choosing a topic:
\begin{itemize}
\item A good thesis topic will simultaneously express a personal vision and participate in a conversation with the literature.
\item Your topic must be one you are passionate about. Nothing less will keep you going. Your personal vision is your reason for being a scientist, an image or principle or idea or goal you care deeply about. It can take many forms. Maybe you want to build a computer you can talk to. Maybe you want to save the world from stupid uses of computers. Maybe you want to demonstrate the unity of all things. Maybe you want to found colonies in space. A vision is always something big. Your thesis can’t achieve your vision, but it can point the way.
\item At the same time, science is a conversation. An awful lot of good people have done their best and they’re written about it. They’ve accomplished a great deal and they’ve completely screwed up. They’ve had deep insights and they’ve been unbelievably blind. They’ve been heros and cowards. And all of this at the same time. Your work will be manageable and comprehensible if it is framed as a conversation with these others. It has to speak to their problems and their questions, even if it’s to explain what’s wrong with them.  A thesis topic that doesn’t participate in a conversation with the literature will be too big or too vague, or nobody will be able to understand it.
\item The hardest part is figuring out how to cut your problem down to a solvable size while keeping it big enough to be interesting. “Solving AI breadth-first” is a common disease; you’ll find you need to continually narrow your topic.  Choosing a topic is a gradual process, not a discrete event, and will continue up to the moment you declare the thesis finished. Actually solving the problem is often easy in comparison to figuring out what exactly it is. If your vision is a fifty-year project, what’s the logical ten-year subproject, and what’s the logical one-year subproject of that? If your vision is a vast structure, what’s the component that gets most tellingly to its heart, and what demonstration would get most tellingly to the heart of that component?
\item An ideal thesis topic has a sort of telescoping organization. It has a central portion you are pretty sure you can finish and that you and your advisor agree will meet the degree requirements. It should have various extensions that are successively riskier and that will make the thesis more exciting if they pan out. Not every topic will fit this criterion, but it’s worth trying for.
\item Some people find that working on several potential thesis projects at once allows them to finish the one that works out and abandon the ones that fail.
\item Topics can be placed in a spectrum from flakey to cut-and-dried. Flakier theses open up new territory, explore previously unresearched phenomena, or suggest heuristic solutions to problems that are known to be very hard or are hard to characterize. Cut-and-dried theses rigorously solve wellcharacterized problems. Both are valuable; where you situate yourself in this spectrum is a matter of personal style.
\item The “further work” sections of papers are good sources of thesis topics.
\end{itemize}


be able to explain simply how each part of your theory
and implementation is in service of the goal.

Make sure once you’ve selected a topic that you get a clear understanding with
your advisor as to what will constitute completion.

Try a simplified version of the thesis problem first. Work examples. Thoroughly
explore some concrete instances before making an abstract theory

There are a number ways you can waste a lot of time during the thesis. Some
activities to avoid (unless they are central to the thesis): language design, userinterface
or graphics hacking, inventing new formalisms, overoptimizing code, tool
building, bureaucracy. Any work that is not central to your thesis should be
minimized.
% ====================
\section{How to do Resarch Methodology}
% ====================

% ====================
\section{Emotional Factors in the Research Process}
% ====================

The entire section by Chapman~\cite{chapman1988how} is worth a read.

Biograpahies: Gregory Bateson’s Advice to a Young Scientist, Freeman Dyson’s
Disturbing the Universe, Richard Feynmann’s Surely You Are Joking, Mr.
Feynmann!, George Hardy’s A Mathematician’s Apology, and Jim Watson’s The Double
Helix
