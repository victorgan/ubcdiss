\chapter{Career}
\label{ch:career}

% ====================
\section{Questions for Job Fairs}
% ====================
As a busy student, how do you keep up to date with technology developments in industry? 
What was the last technology that you heard of and thought 'that's really cool'? 'uncool'? 
Suppose you're locked in a building and can't leave. How many ways can you think of to measure the exterior temperature? 
Resume related questions: Fairtunes, C\&O, Waterloo. 
How can you prevent deadlock in a concurrent system? 
Write a function to add 64 bit numbers using 32bit arithmetic. 
What is a thread? What is a process? 
Write a function that, given an alphabetic string, outputs the characters a) sorted b) duplicates removed and c) in lower case. Do it using only a few (<26) bytes of storage. (eg: "Mississippi" -> "imps") 
Insert into a circular doubly linked list. 
A square island, with side length n meters, lies in the center of a square pond, with side length n+2 meters. Given only two boards of length 0.99 meters, how can you reach the island from the mainland? 
ASCII strings are typically encoded using 1 byte per character, even though for alphanumeric text, only the first 7 bits are used (the MSB is always 0). Write a function that compresses a c string by outputting a string of bytes where each group of seven bits represents each character of input. For example, the 3 byte string 01111111, 01011100, 00000101 would compress to the 21 bit string 111111110111000000101. 
itoa() (He first suggested "reverse the word ordering in a string", then malloc()/free()) 
Tell me about a time your beliefs were challenged in the workplace. 
Tell me about a time when you discovered a fundamental error in the design of your project. 
Design and implement the "shuffle" function for a 200 disc CD changer. Your only constraint is that no song may be repeated until every other song has been played once. 

% ====================
\section{Questions for Interviews}
% ====================

% ====================
\section{Job Fair}
% ====================

Look professional, smile, clear and confident-sounding voice, a firm handshake
and good eye contact. Wear my blue shirt, perhaps get it tailored.

\paragraph{Introduction} Hi, I'm Victor. I'm a second year Master's student in Computer Science doing
research in Robotics and Computer Vision. I'll be looking for full-time
positions starting in September. Does [company] have any opportunities in
software development?

Questions to learn about company culture, organization:
\begin{itemize}
    \item What's pretty hot to know nowadays in Microsoft? What programming
    languages?
    \item Are there code reviews?
    \item On your 'About us' section of your webpage you mention X.
    \item I printed out a copy of a job description on your website. What do you
    mean by 'you need X?'
    \item How do you like your team? Can new hires meet potential teams?
\end{itemize}

\paragraph{Follow up email} Hi John, it was greating speaking with you at the
UBC Technical Career Fair last Wednesday. Thanks for telling me about the
machine learning positions at Microsoft. As mentioned, I'll be looking for
full-time positions starting in September and the positions you've describe seem
like a good fit for me. Here's a copy of the resume I gave you. Feel free to
reach me at 2369997236 or this email address, vhg@cs.ubc.ca.


% ====================
\section{Misc Resources}
% ====================

\url{https://www.cs.ubc.ca/students/undergrad/careers/finding/preparing-technical-career-fair-tips}

\url{http://www.reddit.com/r/cscareerquestions/comments/20ahfq/heres_a_pretty_big_list_of_programming_interview/}:
\blockquote{You probably could get away with a single page cheatsheet but beyond that you probably will be spending more time looking things up and that will count against you. The questions are also very academic in nature but still application oriented so you need to know when, where, and how to use all the different data structures and quickly adapt them to your needs in the question. They will then ask you about performance and a bunch of other things(at least in my interviews) so in order to have a complete cheat sheet you need to have or remember everything you probably learned in your first 2 years of CS classes feeding on your last 2 years of classes. The recruiter I talked to said people at Google said that the phone interview and the process in general was harder than the hardest test they had in school but very similar to a final exam for one of their classes.
}


• Be well prepared, study hard.
• Buy a few good books
• Cracking the Coding Interview
• Programming Interviews Exposed
• How Would You Move Mount Fuji?
• Talk to friends.
• Aggressively pursue interview opportunities. They rarely come to you.
• Make sure you can perform well with little to no sleep.
• Know what they want. Then show you have that.
• Enjoy the game.
• Contact HR if you haven’t heard from them (see JC Oct 6).
• Be enthusiastic.
• No two interview experiences are alike. Prepare for anything and everything.
• Research shows that the interviewer makes up their mind in the first thirty seconds that they meet you.
• Don’t have a cover letter.
• Be honest with yourself and your interviewers.



% ====================
\section{Interview Questions}
% ====================
\subsection{How to Answer Interview Questions}
\blockquote{As an interviewer, I would expect you to know proper syntax for
common things in whatever language you work in (presumably Java), as well as
very common API (substring, working with collections, etc). If you can't write a
for loop, then that would be problematic. Don't know the try-with-resources
syntax? That's OK. If you don't know that substring exists, problem. If you use
"put" instead of "add" (or vice versa) for a collection - not a big deal, that's
what IDEs are for. And even if you mess up on something basic, if you still did
fine on the rest of it, I'll overlook that.}

Algorithm:
\begin{enumerate}
\item Ask clarifying questions
\item Give an example and verify: simple example
\item Give an example and verify: corner case
\item Think about possible solutions and complexity
\item Lower bound complexity
\item Start writing
\item Walk through code
\item Test with: Simple example (walk through code exactly)
\item Test with: b. Corner cases
\end{enumerate}

\subsection{General}
\begin{itemize}
\item Find the most frequent integer in an array
\item Find pairs in an integer array whose sum is equal to 10 (bonus: do it in linear time)
\item Given 2 integer arrays, determine of the 2nd array is a rotated version of the 1st array. Ex. Original Array A={1,2,3,5,6,7,8} Rotated Array B={5,6,7,8,1,2,3}
\item Write fibbonaci iteratively and recursively (bonus: use dynamic programming)
\item Find the only element in an array that only occurs once.
\item Find the common elements of 2 int arrays
\item Implement binary search of a sorted array of integers
\item Implement binary search in a rotated array (ex. {5,6,7,8,1,2,3})
\item Use dynamic programming to find the first X prime numbers
\item Write a function that prints out the binary form of an int
\item Implement parseInt
\item Implement squareroot function
\item Implement an exponent function (bonus: now try in log(n) time)
\item Write a multiply function that multiples 2 integers without using *
\item HARD: Given a function rand5() that returns a random int between 0 and 5, implement rand7()
\item HARD: Given a 2D array of 1s and 0s, count the number of "islands of 1s" (e.g. groups of connecting 1s)
\end{itemize}
\subsection{Strings}
\begin{itemize}
\item Find the first non-repeated character in a String
\item Reverse a String iteratively and recursively
\item Determine if 2 Strings are anagrams
\item Check if String is a palindrome
\item Check if a String is composed of all unique characters
\item Determine if a String is an int or a double
\item HARD: Find the shortest palindrome in a String
\item HARD: Print all permutations of a String
\item HARD: Given a single-line text String and a maximum width value, write the function 'String justify(String text, int maxWidth)' that formats the input text using full-justification, i.e., extra spaces on each line are equally distributed between the words; the first word on each line is flushed left and the last word on each line is flushed right
\end{itemize}
\subsection{Trees}
\begin{itemize}
\item Implement a BST with insert and delete functions
\item Print a tree using BFS and DFS
\item Write a function that determines if a tree is a BST
\item Find the smallest element in a BST
\item Find the 2nd largest number in a BST
\item Given a binary tree which is a sum tree (child nodes add to parent), write an algorithm to determine whether the tree is a valid sum tree
\item Find the distance between 2 nodes in a BST and a normal binary tree
\item Print the coordinates of every node in a binary tree, where root is 0,0
\item Print a tree by levels
\item Given a binary tree which is a sum tree, write an algorithm to determine whether the tree is a valid sum tree
\item Given a tree, verify that it contains a subtree.
\item HARD: Find the max distance between 2 nodes in a BST.
\item HARD: Construct a BST given the pre-order and in-order traversal Strings
\end{itemize}
\subsection{Stacks, Queues, and Heaps}
\begin{itemize}
\item Implement a stack with push and pop functions
\item Implement a queue with queue and dequeue functions
\item Find the minimum element in a stack in O(1) time
\item Write a function that sorts a stack (bonus: sort the stack in place without extra memory)
\item Implement a binary min heap. Turn it into a binary max heap
\item HARD: Implement a queue using 2 stacks
\end{itemize}
\subsection{Linked Lists}
\begin{itemize}
\item Implement a linked list (with insert and delete functions)
\item Find the Nth element in a linked list
\item Remove the Nth element of a linked list
\item Check if a linked list has cycles
\item Given a circular linked list, find the node at the beginning of the loop. Example: A-->B-->C --> D-->E -->C, C is the node that begins the loop
\item Check whether a link list is a palindrome
\item Reverse a linked list iteratively and recursively
\end{itemize}
\subsection{Sorting}
\begin{itemize}
\item Implement bubble sort
\item Implement selection sort
\item Implement insertion sort
\item Implement merge sort
\item Implement quick sort
\end{itemize}

\subsection{Dynamic Programming}
From \url{http://en.wikipedia.org/wiki/Dynamic_programming#Algorithms_that_use_dynamic_programming}:
\begin{itemize}
\item Recurrent solutions to lattice models for protein-DNA binding
\item Backward induction as a solution method for finite-horizon discrete-time dynamic optimization problems
\item Method of undetermined coefficients can be used to solve the Bellman equation in infinite-horizon, discrete-time, discounted, time-invariant dynamic optimization problems
\item Many string algorithms including longest common subsequence, longest increasing subsequence, longest common substring, Levenshtein distance (edit distance)
\item Many algorithmic problems on graphs can be solved efficiently for graphs of bounded treewidth or bounded clique-width by using dynamic programming on a tree decomposition of the graph.
\item The Cocke–Younger–Kasami (CYK) algorithm which determines whether and how a given string can be generated by a given context-free grammar
\item Knuth's word wrapping algorithm that minimizes raggedness when word wrapping text
\item The use of transposition tables and refutation tables in computer chess
\item The Viterbi algorithm (used for hidden Markov models)
\item The Earley algorithm (a type of chart parser)
\item The Needleman–Wunsch and other algorithms used in bioinformatics, including sequence alignment, structural alignment, RNA structure prediction
\item Floyd's all-pairs shortest path algorithm
\item Optimizing the order for chain matrix multiplication
\item Pseudo-polynomial time algorithms for the subset sum and knapsack and partition problems
\item The dynamic time warping algorithm for computing the global distance between two time series
\item The Selinger (a.k.a. System R) algorithm for relational database query optimization
\item De Boor algorithm for evaluating B-spline curves
\item Duckworth–Lewis method for resolving the problem when games of cricket are interrupted
\item The value iteration method for solving Markov decision processes
\item Some graphic image edge following selection methods such as the "magnet" selection tool in Photoshop
\item Some methods for solving interval scheduling problems
\item Some methods for solving word wrap problems
\item Some methods for solving the travelling salesman problem, either exactly (in exponential time) or approximately (e.g. via the bitonic tour)
\item Recursive least squares method
\item Beat tracking in music information retrieval
\item Adaptive-critic training strategy for artificial neural networks
\item Stereo algorithms for solving the correspondence problem used in stereo vision
\item Seam carving (content aware image resizing)
\item The Bellman–Ford algorithm for finding the shortest distance in a graph
\item Some approximate solution methods for the linear search problem
\item Kadane's algorithm for the maximum subarray problem
\end{itemize}
