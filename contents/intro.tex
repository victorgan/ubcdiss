%% The following is a directive for TeXShop to indicate the main file
%%!TEX root = diss.tex

% Describe the problem
% State your contributions (write the list of contributions fist)
% that’s all
% use an example to introduce the problem
% Reader thinks“gosh, if they can really deliver this, that’d be exciting; Id
% bet- ter read on” [? ]
% Contributions should be refutable
% No ”the rest of the paper is”. Instead, use forward references from the nar-
% rative in the introduction. The introduction (including the contributions)
% should survey the whole paper, and therefore forward reference every im-
% portant part

\chapter{Introduction}
\label{ch:Introduction}
% % ====================
% \section{Summary of Thesis}
% % ====================
% % --------------------
% \subsection{One sentence summary}
% % --------------------
% 
% % --------------------
% \subsection{One paragraph summary}
% % --------------------
% 
% % --------------------
% \subsection{Five minute read summary}
% % --------------------




(Contributions) In this paper we put this point on a further basis:
\begin{itemize}
\item contribution 1
\end{itemize}

\endinput

Any text after an \endinput is ignored.
You could put scraps here or things in progress.

\paragraph{What is this?} A collection of research ideas.

\paragraph{Who's it for?} Me.
% Purpose: Motivate the need of an autonomous back-in parking system.

Mobility impairment is a big problem.
Lack of mobility leads to social, psychological and physical problems.
Powered wheelchairs can improve mobility for those unable to use manual wheelchairs.

% Smart wheelchairs are needed.
However, it may be impossible persons with severe disabilities to maneouver PWCs
given their current control interfaces: 10\% of patients who receive PWC traning
find it extremely difficult or impossible to use the whelchair for activities of
daily living \cite{fehr2000adequacy}.

Moreso, when asked specifically about steering and maneuvering tasks, the
percentage of patients reported to find these difficult or impossible jumped to
40, and  nearly half of patients unable to control a power wheelchair by
conventional methods would benefit from an automated navigation system,
according to the clinicians who treat them. 
Due to this, Fehr et al. \cite{fehr2000adequacy} concludes supervised autonomous
navigation is necessary over more innovative steering interfaces.

% The potential impact for smart wheelchairs is significant.
An estimated 1.4 million to 2.1 million individuals in 2008 stood to benefit
from smart wheelchairs, and an estimated 973 thousand to 1.7 million of those
stood to benefit from autonomous navigation capabilities in particular. These
numbers have been expected to grow proportional to the increased prevalance of
wheelchair users at a rate of 5.9 percent per year \cite{simpson2008many}.

% TODO Pouria's intro, page 2 - 6. Rest of Pouria

% TODO talk about shared control? Pouria's thesis, page 1
% However, total autonomy is hard and even detrimental.
% A system that completes specific hard tasks, akin to automatic parallel
% parking or cruise control on a car, is preferred.

\section{Previous work on Smart Wheelchairs}
