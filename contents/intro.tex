%% The following is a directive for TeXShop to indicate the main file
%%!TEX root = diss.tex

% Describe the problem
% State your contributions (write the list of contributions fist)
% that’s all
% use an example to introduce the problem
% Reader thinks“gosh, if they can really deliver this, that’d be exciting; Id
% bet- ter read on” [? ]
% Contributions should be refutable
% No ”the rest of the paper is”. Instead, use forward references from the nar-
% rative in the introduction. The introduction (including the contributions)
% should survey the whole paper, and therefore forward reference every im-
% portant part

\chapter{Introduction}
\label{ch:Introduction}




\section{The Impact of Autonomous Wheelchairs}

% Purpose: Motivate the need of an autonomous back-in parking system.

With the number of US citizens over the age of 65 expected to grow by 75\% by
2030 \cite{simpson2008many}, ensuring older adults maintain their sense of
mobility becomes increasingly important.
For residents in long-term care facilities, independent mobility is directly
related to their quality of life and sense of freedom, choice and independence
\cite{bourret2002meaning}.
For people of any age, loss of independent mobilitiy leads to feelings of
emotional loss, reduced-self esteem, isolation, stress and fear of abandonment
\cite{finlayson2003experiencing}, and decreases opportunities to socialize that
lead to anxiety and depression \cite{iezzoni2001mobility}. 

Wheelchairs offer a solution, but many users find operating existing manual or
powered wheelchairs (PWCs) difficult or impossible \cite{simpson2008many}.
10\% of patients who receive PWC training find it extremely difficult or
impossible to use the wheelchair for activities of daily living, and
when asked specifically about steering and maneuvering tasks, the
percentage jumps to 40\% \cite{fehr2000adequacy}.
Fehr et al. \cite{fehr2000adequacy} concludes improving steering interfaces is
not enough due to the type of impairments the users face, and autonomous
navigation is necessary. Half of patients unable to normally control a powered
wheelchair would benefit from an automated navigation system.
This population that stands to benefit is significant:
In 2008, 1.4 to 2.3 million people in the US were estimated to
benefit from powered wheelchairs with some autonomous functions, and of those
973 thousand to 1.7 million will benefit particularly from autonomous
navigation capabilities with a growth rate of 5.9\% per year \cite{simpson2008many}.
Powered wheelchairs (PWCs) with autonomous maneuvering capabilities is both
necessary and will directly improve the quality of life of millions of people.

Complete PWC automation, however, is not suitable. 
A totally autonomous system may deprive user's feeling of being in control 
\cite{viswanathana2014wizard}.
Older adults preferred robot assistance over human assistance for tasks related
to chores, manipulating objects, and information management, but not personal
care and leisure activities \cite{smarr2014domestic}.
Instead, task-specific autonomous routines is preferable, akin to automatic
parallel parking or cruise control in a car.
Some work has been done on task-specific autonomous PWC behaviours, including
navigation in crowded environments \cite{prassler2001robotics}, obstacle
avoidance \cite{viswanathan2012navigation}, and docking onto lift platforms
\cite{sermeno2006vision} and custom-designed beds \cite{ren2012docking}.
More general overviews of smart wheelchair systems have also been described
\cite{viswanathan2012navigation, simpson2005smart, faria2013patient}. 

One common task is back-in parking.
% Not much in general back-in parking, a very common task in the target demographic
Little work has been done to dock/back-in park a wheelchair in a more general
environment, such a general indoor office environment or a room in a
long-term care facility. 

This motivates the work in this thesis: we focus on automating the back-in
parking task for a PWC, a specific task that has been proven desirable and
helpful to be automated.

\section{Problem Definition}
How does a machine command a wheelchair to back-in park? We decompose this into two
tasks, identifying a parking spot and moving into the identified parking spot,
for modularity and for the solutions to be transferable to a large range of
topics.
Identifying a parking spot begs many questions:
\begin{itemize}
\item How do you define a parking spot?
\item How do you determine feasible parking spots around you?
\item How do you predict which parking spot the user has in mind, if there is any?
\item What makes some spots better than others, and in what situations?
\item What has already been done?
\item How can one answer these questions in a solution that is 
computationally efficient or even real-time?
\end{itemize}

In this thesis I focus on two of these questions: how do you define a place to
park on an open floorspace, how do you back into said parking spot.

(Contributions) In this paper we put this point on a further basis:
\begin{itemize}
\item Introduce a pipeline that picks a place to park
\item Introduce an efficient algorithm to perform (+,min)-convolutions
\item Validates algorithm with empirical performance measures
\item Introduce a pipeline that backs-in a wheelchair
\item Validates the algorithm with real world experimental results
\end{itemize}

Chapter X describes background and preliminary notation used.

\section{Hardware}
We choose and fix the hardware used beforehand.

We use an RGB-D sensor due to its suitability to the problem:
it works well indoors, where the target users will be using the wheelchairs; it
is cost effective; it provices depth information with adequete accuracy.

We use only one camera. This is for simplicity. Each camera saturates its USB
controller, hence a computer must have as many USB controllers as cameras.
An additional calibration step to register the information, increasing the
computational costs even further than the linear factor. 
One can argue a limitation of using one RGB-D sensor is the depth measurements
have a range of around 0.6 m to 8 m, with a field of view of 43 degrees
vertically by 57 degrees horizontally (TODO better reference) \cite{endres2014catadioptric},
comparitively limited to lidars with fields of view of 180 degrees. Some work
\cite{endres2014catadioptric} has been done using two mirrors to split the field
of view of a single camera to cover both the front and rear view. Our work
sticks to a single camera for simplicity, though it can be naturally extended.

We mount the RGB-D camera downwards. However, we do not rely on a specific angle
to keep the solutions generalizable to systems that might have different camera
angles. The only assumption is the ground plane is visible.

We use an Asus RGB-D camera as it is smaller and less obtrustive than the Kinect
and does not require an additional power plug; it gets its power entirely from
USB.


We use a (TODO model) by (TODO) 

We use a Lenovo W530 laptop with 
\begin{itemize}
\item Intel Core i7-3720QM Processor
\item 8GB RAM
\item 120GB SSD
\item NVIDIA Quadro K1000M Graphical Processing Unit
\item Two dedicated USB controllers for USB 2.0 and USB 3.0
\item Ubuntu 14.04 64-bit
\item Robotic Operating System (Indigo release)
\end{itemize}


% ======================
\endinput
Any text after an \endinput is ignored.
% ======================

% ====================
\section{Summary of Thesis}
% ====================
% --------------------
\subsection{One sentence summary}
% --------------------
We introduce an autonomous back-in parking system for powered wheelchairs
equipped with an RGB-D sensor, that finds a suitable place to park in a room
without any marked parking spots.

% --------------------
\subsection{One paragraph summary}
% --------------------
We address the problem: Where

% --------------------
\subsection{Five minute read summary}
% --------------------

\paragraph{What is this?} A collection of research ideas.

\paragraph{Who's it for?} Me.

TODO use this \cite{viswanathan2011navigation}
