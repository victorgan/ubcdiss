%% The following is a directive for TeXShop to indicate the main file
%%!TEX root = diss.tex

% - Describe the problem
% - State your contributions (write the list of contributions fist)
% - that's all

% use an example to introduce the problem
% Reader thinks“gosh, if they can really deliver this, that’d be exciting; Id
% bet- ter read on” [? ]
% Contributions should be refutable
% No ”the rest of the paper is”. Instead, use forward references from the nar-
% rative in the introduction. The introduction (including the contributions)
% should survey the whole paper, and therefore forward reference every im-
% portant part

\chapter{Introduction}
\label{ch:Introduction}

Consider these situations:
\begin{enumerate}
\item There is a gathering of wheelchair users in a ballroom of a long-term care
facility. A user wishes to park its wheelchair so there is still room for other
wheelchair users and for an entertaining event to happen in the centre of the
floor.
\item A car must decide where to park in an unmarked lot -- perhaps the lot is
grass, dirt or snow-covered.
\item A general 2D packing problem, where we would like to pack as many
arbitrarily-shaped objects into a container that itself is arbirarily shaped. We
are able to rotate the objects to any angle.
\end{enumerate}

In all three situations, where would one place the wheelchair, car or object in
a given area? In each case, the exact criteria to consider may vary, but many
similarities exist.  Good placements may accomodate room for subsequent
wheelchairs, cars and objects. Perhaps objects are best placed so they do not
block previously placed objects from being able to leave the space. Perhaps
objects placements are best when packed tightly into spaces, locally aligned
with objects around it, minimizing space in the cracks between objects. And,
perhaps placements should be as close to the object's current position as
posssible to minimize wheeling effort or fuel cost.

This thesis address the problem of how to autonomously decide where to
place an object in an open floorspace, with specific application to the first
situation mentioned: identifying suitable parking spots for back-in parking of
powered wheelchairs in long term care facilities. 
\autoref{ch:background} describes the background of the problem in more detail.
Specifically, my contributions are:

\begin{itemize}
\item A formulation of a value function that aims to characterize the robot's
unobtrusiveness, which is used determine where to place robots on an obstacle
map.  This is introduced in \autoref{ch:identify}.
\item An efficient linear-time algorithm of the value function that generalizes to
performing $(+,\min)$-convolutions over two distance transforms, also seen in \autoref{ch:identify}.
\item A real-world application using the value function as a goal state for a
back-in parking task with an RGBD camera and inexact control models,
as shown in \autoref{ch:wheelchair}.
\end{itemize}



% ======================
\endinput
Any text after an \endinput is ignored.
% ======================

% ====================
\section{Summary of Thesis}
% ====================
% --------------------
\subsection{One sentence summary}
% --------------------
We introduce an autonomous back-in parking system for powered wheelchairs
equipped with an RGB-D sensor, that finds a suitable place to park in a room
without any marked parking spots.

% --------------------
\subsection{One paragraph summary}
% --------------------
We address the problem: Where

% --------------------
\subsection{Five minute read summary}
% --------------------

\paragraph{What is this?} A collection of research ideas.

\paragraph{Who's it for?} Me.

TODO use this \cite{viswanathan2011navigation}



\item Introduce a pipeline that picks a place to park
\item Introduce an efficient algorithm to perform (+,min)-convolutions
\item Validates algorithm with empirical performance measures
\item Introduce a pipeline that backs-in a wheelchair
\item Validates the algorithm with real world experimental results
