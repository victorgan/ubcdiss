%% The following is a directive for TeXShop to indicate the main file
%%!TEX root = diss.tex

% - Describe the problem
% - State your contributions (write the list of contributions fist)
% - that's all

% use an example to introduce the problem
% Reader thinks“gosh, if they can really deliver this, that’d be exciting; Id
% bet- ter read on” [? ]
% Contributions should be refutable
% No ”the rest of the paper is”. Instead, use forward references from the nar-
% rative in the introduction. The introduction (including the contributions)
% should survey the whole paper, and therefore forward reference every im-
% portant part

\chapter{Introduction}
\label{ch:Introduction}




\section{The Impact of Autonomous Wheelchairs}

% Purpose: Motivate the need of an autonomous back-in parking system.

With the number of US citizens over the age of 65 expected to grow by 75\% by
2030 \cite{simpson2008many}, ensuring older adults maintain their sense of
mobility becomes increasingly important.
For residents in long-term care facilities, independent mobility is directly
related to their quality of life and sense of freedom, choice and independence
\cite{bourret2002meaning}.
For people of any age, loss of independent mobilitiy leads to feelings of
emotional loss, reduced-self esteem, isolation, stress and fear of abandonment
\cite{finlayson2003experiencing}, and decreases opportunities to socialize that
lead to anxiety and depression \cite{iezzoni2001mobility}. 

Wheelchairs offer a solution, but many users find operating existing manual or
powered wheelchairs (PWCs) difficult or impossible \cite{simpson2008many}.
10\% of patients who receive PWC training find it extremely difficult or
impossible to use the wheelchair for activities of daily living, and
when asked specifically about steering and maneuvering tasks, the
percentage jumps to 40\% \cite{fehr2000adequacy}.
Fehr et al. \cite{fehr2000adequacy} concludes improving steering interfaces is
not enough due to the type of impairments the users face, and autonomous
navigation is necessary. Half of patients unable to normally control a powered
wheelchair would benefit from an automated navigation system.
This population that stands to benefit is significant:
In 2008, 1.4 to 2.3 million people in the US were estimated to
benefit from powered wheelchairs with some autonomous functions, and of those
973,000 to 1.7 million will benefit particularly from autonomous
navigation capabilities with a growth rate of 5.9\% per year \cite{simpson2008many}.
Powered wheelchairs (PWCs) with autonomous maneuvering capabilities are both
necessary and will directly improve the quality of life of millions of people.

Complete PWC automation, however, is not suitable. 
A totally autonomous system may deprive a user's feeling of being in control 
\cite{viswanathana2014wizard}.
Older adults preferred robot assistance over human assistance for tasks related
to chores, manipulating objects, and information management, but not personal
care and leisure activities \cite{smarr2014domestic}.
Instead, task-specific autonomous routines is preferable, akin to automatic
parallel parking or cruise control in a car.
Some work has been done on task-specific autonomous PWC behaviours, including
navigation in crowded environments \cite{prassler2001robotics}, obstacle
avoidance \cite{viswanathan2012navigation}, and docking onto lift platforms
\cite{sermeno2006vision} and custom-designed beds \cite{ren2012docking}.
More general overviews of smart wheelchair systems have also been described
\cite{viswanathan2012navigation, simpson2005smart, faria2013patient}. 

One common task is back-in parking [TODO bikram thesis/pooja's draft]. 
% Not much in general back-in parking, a very common task in the target demographic
Little work has been done to dock/back-in park a wheelchair in a more general
environment, such as an indoor office or a room in a long-term care facility. 

This motivates the application of the work in this thesis:
we focus on automating the back-in parking task for a PWC, a specific task that
has been proven desirable and helpful to be automated.

% \section{Motion Planning with Ill-Defined Goal States}
% 
% Lots of work has been done to determine a path given a goal state/set. This
% assumes a goal set has been previously defined.
% 
% Lots of work has been done on determining objects, scenes. But little work has
% been done to combine the two: Determinine a suitable goal state that takes into
% account feasibility and choosing a goal state that is present in the scene. 
% 
% This is closely related to developing a potential function: a pseudo-metric that
% defines how close you are to a goal. Our work skips the 'given a goal state,
% design a potential function' step, and develops a potential field as a
% by-product of generating goal states.
% 
% Often it makes sense to have a fixed goal set, or one goal state. If your robot
% wants to reach the fridge, having the location of the fridge is an appropriate
% goal.
% We do away with the notion that there is a fixed goal set. Often the goal set is
% fuzzy, we do not know much about it. Instead, we form a desirability function
% (which can be seen as a negative potential field) over the configuration space.
% We then plan to reach a high level in the desirability function given certain
% constraints.
% 
% Take, for example, the problem of placing a robot to rest for a night. Where
% should you place it? In most cases there isn't a correct answer. There is no
% well-defined goal state. There are also constraints that must be satisfied:
% obstacle avoidance, minimum path length. Instead of posing the problem as path
% planning towards an arbitrary goal state, we pose it as steering the robot to an
% incrementally better configuration.
% 
% It's often assumed you can see what your goal state is like. There are a lot of
% times a goal state isn't defined by what it looks like, but by the context
% surrounding it: perhaps it is an enclosed space that fits the robot just right.


\section{Problem Definition}
How does a machine command a wheelchair to back-in park? We decompose this into two
tasks, identifying a parking spot and moving into the identified parking spot,
for modularity and for the solutions to be transferable to a large range of
topics.
Identifying a parking spot begs many questions:
\begin{itemize}
\item How do you define a parking spot?
\item How do you determine feasible parking spots around you?
\item How do you predict which parking spot the user has in mind, if there is any?
\item What makes some spots better than others, and in what situations?
\item What has already been done?
\item How can one answer these questions in a solution that is 
computationally efficient or even real-time?
\end{itemize}

In this thesis I focus on two of these questions: how do you define a place to
park on an open floorspace, how do you back into said parking spot.

(Contributions) In this paper we put this point on a further basis:
\begin{itemize}
\item Introduce a pipeline that picks a place to park
\item Introduce an efficient algorithm to perform (+,min)-convolutions
\item Validates algorithm with empirical performance measures
\item Introduce a pipeline that backs-in a wheelchair
\item Validates the algorithm with real world experimental results
\end{itemize}

Chapter X describes background and preliminary notation used.

Often determining the goal state, localization/mapping, planning and movement
are all done separately. We introduce an all-in-one solution.



% ======================
\endinput
Any text after an \endinput is ignored.
% ======================

% ====================
\section{Summary of Thesis}
% ====================
% --------------------
\subsection{One sentence summary}
% --------------------
We introduce an autonomous back-in parking system for powered wheelchairs
equipped with an RGB-D sensor, that finds a suitable place to park in a room
without any marked parking spots.

% --------------------
\subsection{One paragraph summary}
% --------------------
We address the problem: Where

% --------------------
\subsection{Five minute read summary}
% --------------------

\paragraph{What is this?} A collection of research ideas.

\paragraph{Who's it for?} Me.

TODO use this \cite{viswanathan2011navigation}
