%% The following is a directive for TeXShop to indicate the main file
%%!TEX root = diss.tex

% Describe the problem
% State your contributions (write the list of contributions fist)
% that’s all
% use an example to introduce the problem
% Reader thinks“gosh, if they can really deliver this, that’d be exciting; Id
% bet- ter read on” [? ]
% Contributions should be refutable
% No ”the rest of the paper is”. Instead, use forward references from the nar-
% rative in the introduction. The introduction (including the contributions)
% should survey the whole paper, and therefore forward reference every im-
% portant part

\chapter{Introduction}
\label{ch:Introduction}




\section{The Impact of Autonomous Back-in Parking}

% Purpose: Motivate the need of an autonomous back-in parking system.

% Lack of mobility leads to social, psychological and physical problems.
Mobility impairment is a significant problem for older adults. Mobility is
shown to be pivotal to the quality of life of residents in long-term care
facilities, offering a means of freedom, choice and independence
\cite{bourret2002meaning}.

% The potential impact for smart wheelchairs is significant.
Smart wheelchairs have a large potential for impact.
One method in aiding mobility is using powered wheelchairs with additional smart
capabilities.
An estimated 1.4 million to 2.1 million individuals in the US in 2008 stood to
benefit from smart wheelchairs, and an estimated 973 thousand to 1.7 million of
those stood to benefit from autonomous navigation capabilities in particular.
These estimates are expected to grow proportional to the increase in
wheelchair users, which grows at a rate of 5.9 percent per year
\cite{simpson2008many}.

% Smart wheelchairs are needed.
However, it may be impossible for persons with severe disabilities to maneuver PWCs
given their current control interfaces: 10\% of patients who receive PWC training
find it extremely difficult or impossible to use the wheelchair for activities of
daily living \cite{fehr2000adequacy}.
Moreso, when asked specifically about steering and maneuvering tasks, the
percentage of patients reported to find these difficult or impossible jumped to
40\%, and nearly half of patients unable to control a power wheelchair by
conventional methods would benefit from an automated navigation system.
Due to this, Fehr et al. \cite{fehr2000adequacy} concludes supervised autonomous
navigation is necessary over improved steering interfaces.

% Semi-automation is the way to go
Though autonomous tasks are necessary, total autonomy is hard and may not even be desired. 
A totally autonomous system may deprive user's feeling of being in control 
\cite{viswanathana2014wizard}.
Older adults preferred robot assistance over human assistance for tasks related
to chores, manipulating objects, and information management, but not personal
care and leisure activities \cite{smarr2014domestic}.
Instead, systems that complete specific chore-like tasks akin to automatic
parallel parking or cruise control on a car is more suitable.

% There has been some progress on semi-automation
This leads to the need of task-specific autonomous behaviours.
Some work has been done on task-specific autonomous PWC behaviours, including
navigation in crowded environments \cite{prassler2001robotics}, obstacle
avoidance \cite{viswanathan2012navigation}, and docking onto lift platforms
\cite{sermeno2006vision} and custom-designed beds \cite{ren2012docking}.
More general overviews of smart wheelchair systems can be found in
\cite{viswanathan2012navigation, simpson2005smart, faria2013patient} (TODO
confirm).

% Not much in general back-in parking, a very common task in the target demographic
Little work has been done to dock/back-in park a wheelchair in a more general
environment, such a general indoor office environment akin to a room in a
long-term care facility. 

This motivates the work in this thesis: we focus on automating the back-in
parking task, a specific task that has been proven desirable and helpful to be
automated.

\section{Problem Definition}
How does a machine command a wheelchair to back-in park? We decompose this into two
tasks, identifying a parking spot and moving into the identified parking spot,
for modularity and for the solutions to be transferable to a large range of
topics.
Identifying a parking spot begs many questions:
\begin{itemize}
\item How do you define a parking spot?
\item How do you determine feasible parking spots around you?
\item How do you predict which parking spot the user has in mind, if there is any?
\item What makes some spots better than others, and in what situations?
\item What has already been done?
\item How can one answer these questions in a solution that is 
computationally efficient or even real-time?
\end{itemize}



\section{Hardware}
We choose and fix the hardware used beforehand.

We use an RGB-D sensor due to its suitability to the problem:
it works well indoors, where the target users will be using the wheelchairs; it
is cost effective; it provices depth information with adequete accuracy.

We use only one camera. This is for simplicity. Each camera saturates its USB
controller, hence a computer must have as many USB controllers as cameras.
An additional calibration step to register the information, increasing the
computational costs even further than the linear factor. 
One can argue a limitation of using one RGB-D sensor is the depth measurements
have a range of around 0.6 m to 8 m, with a field of view of 43 degrees
vertically by 57 degrees horizontally (TODO better reference) \cite{endres2014catadioptric},
comparitively limited to lidars with fields of view of 180 degrees. Some work
\cite{endres2014catadioptric} has been done using two mirrors to split the field
of view of a single camera to cover both the front and rear view. Our work
sticks to a single camera for simplicity, though it can be naturally extended.

We mount the RGB-D camera downwards. However, we do not rely on a specific angle
to keep the solutions generalizable to systems that might have different camera
angles. The only assumption is the ground plane is visible.

We use an Asus RGB-D camera as it is smaller and less obtrustive than the Kinect
and does not require an additional power plug; it gets its power entirely from
USB.


We use a (TODO model) by (TODO) 

We use a Lenovo W530 laptop with 
\begin{itemize}
\item Intel Core i7-3720QM Processor
\item 8GB RAM
\item 120GB SSD
\item NVIDIA Quadro K1000M Graphical Processing Unit
\item Two dedicated USB controllers for USB 2.0 and USB 3.0
\item Ubuntu 14.04 64-bit
\item Robotic Operating System (Indigo release)
\end{itemize}

(Contributions) In this paper we put this point on a further basis:
\begin{itemize}
\item contribution 1
\end{itemize}

% ======================
\endinput
Any text after an \endinput is ignored.
% ======================

% ====================
\section{Summary of Thesis}
% ====================
% --------------------
\subsection{One sentence summary}
% --------------------
We introduce an autonomous back-in parking system for powered wheelchairs
equipped with an RGB-D sensor, that finds a suitable place to park in a room
without any marked parking spots.

% --------------------
\subsection{One paragraph summary}
% --------------------
We address the problem: Where

% --------------------
\subsection{Five minute read summary}
% --------------------

\paragraph{What is this?} A collection of research ideas.

\paragraph{Who's it for?} Me.

TODO use this \cite{viswanathan2011navigation}
