% ===================================================
\chapter{Related Work}
% ===================================================

% show my problem is unsolved
% • show my problem is interesting
% • show my idea works
% • show how my idea compares to other ideas
% • No related work yet! Include at end.
% – Problem 1: describing alternative approaches gets between the reader
% and your idea (I feel tired)
% – Problem 2: the reader knows nothing about the problem yet; so your
% (carefully trimmed) description of various technical tradeoffs is abso-
% lutely incomprehensible (I feel stupid)
% – instead, Concentrate single-mindedly on a narrative that
% ∗ Describes the problem, and why it is interesting
% ∗ Describes your idea
% ∗ Defends your idea, showing how it solves the problem, and filling
% out the details
% ∗ On the way, cite relevant work in passing, but defer discussion to
% the end
% • show how my idea compares to other ideas
% • Be generous to the competition. In his inspiring paper [Foo98] Foogle shows....
% We develop his foundation in the following ways...
% • Giving credit to others does not diminish the credit you get from your paper
% • Acknowledge weaknesses in your approach
% • Failing to give credit to others can kill your paper: You don’t know that it’s
% an old idea (bad) or You do know, but are pretending it’s yours (very bad)



% ====================
\section{Parking Lot Detection}
% ====================
Wang \cite{wang2014automatic}
Jung \cite{jung2014semiautomatic}

Pouria's thesis \cite{talebifard2014risk}
Talk: \url{https://www.youtube.com/watch?v=Db7EhLYf-Yk#t=1h11m07s}

% ====================
\section{SLAM}
% ====================

LSD-Slam and PTAM for monocular seem to be state-of-art: \url{http://vision.in.tum.de/research/lsdslam}

review paper: \url{http://www-personal.acfr.usyd.edu.au/tbailey/papers/slamtute2.pdf}

2009 review paper: \url{http://www.le2i.cnrs.fr/IMG/publications/2172_Muhammad_EI_2009.pdf}

% ====================
\section{Collision Avoidance}
% ====================
Old paper using quadtrees\cite{ghoshray1996comprehensive}

Bigdog \cite{raibert2008bigdog}

Springer handbook of robotics \cite{siciliano2008springer}: Vector field histograms, as described in Pouria's thesis \cite{talebifard2014risk}


